\subsection{Issues to Explore}
Once we set up both Junto and Textgrounder, there are a number of hypotheses that we would like to test.

There are a few things to consider for determining the structure of the graph.
The first question is whether the graph should be directed, with edges pointing from referring document to the referred document, or undirected. 

The second set of questions pertain to the method to be used for inducing the edges between documents;
for example, whether the edges are based solely on hyperlinks between nodes, whether the edges are transitive, and whether we should induce a dense graph with many edges between nodes, or a sparse graph with a few strong connections.
Additionally, we could induce an edge between all documents that fall in a particular grid cell to increase locality, or between documents in adjacent cells, to introduce smoothing.

A third challenge is determining the weights of edges.
Intuitively, the weight of an edge should be proportional to the similarity of the two nodes (documents).
This notion of similarity can be formalized in a number of ways.
One approach would be to take the similarity between the language models for the two documents as the edge weights.
This could be the cosine similarity, tf-idf weighted cosine similarity, or the KL divergence score for the pair.
An straightforward alternative to begin with would be to use the same edge weight for all edges regardless of document similarity.
 
Furthermore, since each document can potentially be related to more than one
location, we will examine the performance of holding seed document label
distributions fixed versus allowing them to vary (and acquire non-zero weights
for other locations).
The latter approach would allow a document label to be influenced by related geographic locations; 
for example, the distribution for the article for Lake Austin would end up having a nonzero weight for the location for Travis County.
As suggested earlier, this might be useful in geolocating concepts that span multiple locations.

After further exploration of the dataset and tools, we will systematically test these alternatives, and determine the best models and parameters to use for the label propagation algorithm.
If label propagation give promising results, we can then examine the possibility of augmenting the pseudo-documents in language-models based approaches with the additional information from the new documents geolocated using label propagation.

