\begin{abstract}
    We believe that the link structure of Wikipedia contains information that
    would potentially be useful for geolocation.
    By using label-propagation algorithms to push information about geolocation
    around that link structure, we will be able to explore new possibilities
    for geolocation. 
    For example, we will be able to examine the principles of geolocation on
    non-geotagged articles.
    Furthermore, the labeling model gives us a way to associate text with
    multiple locations, which may provide new information.
    Unfortunately, our progress has been stymied by a host of technical issues,
    which we must overcome in order to build our basic label and graph models.
\end{abstract}

\section{Introduction}
\comment{You can now enclose text in a comment block to comment it out inline}
Geographic information is relevant in several contexts; for example, geographic
information retrieval for exploring document collections
\cite{ding2000computing}, toponym resolution in historical texts
\cite{smith2001disambiguating}, summarizing travelogues and travel
recommendation \cite{hao-et-al:10}, socio-linguistic studies
\cite{eisenstein-smith-xing:11} and targeted advertising on the internet. 
With the widespread use of mobile devices, geographic information is becoming
increasingly ubiquitous and important. 
In this project we propose a method for predicting the geographic location of
Wikipedia articles with a graph based semi-supervised algorithm and a small
amount of labeled data.

Prior work on geolocating Wikipedia articles (\cite{wing-baldridge:11}
\cite{rolleretal:12}) has so far only used a subset of the English language
Wikipedia for training; specifically, it uses those articles that already have
geolocation tags associated with them. 
These articles constitute a relatively small portion of the English language
Wikipedia. \comment{find actual figure} 
This is not surprising, since most articles on Wikipedia are not about
geographic locations, or objects at specific locations.
On the other hand, the number of articles that have links to or from such
geolocated articles is much larger, which can be a potential source of useful
information. \comment{ find actual figure}

Earlier approaches used geotagged article content indirectly by first building
unigram language models from the text of the articles during training, and then
comparing similarity of test documents with these language models.
This ignores the potentially useful information present in the  hyperlinks
between articles, which is a strong indicator of semantic and geographic
relatedness, and hence can be used to improve prediction accuracy.

Another assumption inherent in prior work is that each article is associated
with a single location, i.e. the location specified by the geo-tag.
While this is a fair assumption for some articles, e.g., those about historical
monuments and physical objects, this assumption is not necessarily valid for
articles about geographically distributed concepts such as states (e.g., Texas)
or personal biographies (e.g., George Washington). 

Our hypothesis is that we can infer the location for non-geotagged articles by
using the location information present in other geotagged articles that they
link to, or receive incoming links from.
Unlike prior work, our proposed approach makes use of the link structure of the
articles in addition to the text based language models.
We will use the label propagation algorithm to infer a distribution over
locations for each non-geotagged article starting from a small seed set of
geotagged articles.
This in turn would allow us to include the text from newly geo-tagged articles
in the language models for the locations they are strongly associated with.

The majority of our time so far has been spent dealing with a host of
technical issues that we need to solve in order to set up our preprocessing and
evaluation system.
These issues have prevented us from being able to even set up a simple case of
our proposed model; we have only recently been able to develop method that will
allow us to obtain cell information from Textgrounder, which we will need in
order to define the labels for propagation.
Furthermore, obtaining a proper graph representation of Wikipedia remains a
major technical obstacle for us.
We report on these issues in depth in a section on our current status, and
though we have faced major obstacle, we remain optimistic that we will be able
to achieve our goals.
To that end, we have reviewed and revised our plans for getting results, and we
have updated the section on our next steps to reflect this.


