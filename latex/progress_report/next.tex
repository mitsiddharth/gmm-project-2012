\section{Next steps}
Getting a comparison of label propagation and Textgrounder for geolocation on
the uniform grid remains our primary first step.
In order to do this, we will need to recreate the uniform case of
\cite{rolleretal:12}, which means successfully getting Textgrounder up and
running on TACC.
Once we have a log file, we'll also be able to extract the cell information we
need to create labels for the label propagation system; with that, we can use
the same training set to provide seed labels.
After that, we can extract the predicted locations and (hopefully) supply them
to an existing evaluation mechanism.
Figuring out how to perform these steps on the simple case will allow us to
explore more sophisticated models, both for creating geographic cells (i.e.
using kd-trees), and for constructing the document connection graph.
\par
The primary motivation for this project is still to attempt to extend
geographic labelling to the rest of wikipedia. 
Once we figure out which, if any, label propagation models will work well
enough for this purpose, we can explore the possibilities made available in the
new information from the extended set.
If we can surmount our current technical difficulties and reach this point, we
will be able to explore questions such as how similar are the locational
predictions produced by the IR model vs. the graph-based label-propagation
model, and how good is the information contained in the label-extended portion for
predicting locations in the geotagged portions.

