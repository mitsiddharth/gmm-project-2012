\subsection{Textgrounder: a study in failure}
So far, we have not been successful in using the Textgrounder system of
\cite{wing-baldridge:11} and \cite{rolleretal:12} to accomplish our goals. 
Several issues stand in our way, including structural problems in the
Textgrounder codebase, mistakes and poor decision making in attempting to get
the system up and running, and various problems involving dataset and resource
availability.
\par
Textgrounder is, unfortunately, a research project, and therefore has a messy
codebase.  
While we regret having to say negative things about the products of what
amounts to volunteer efforts, the design of the system presents serious
obstacles to anyone wishing to do further research based on it.
Our objective has been to extract cell assignments for documents in the
Wikipedia corpus, so we could use those as labels for label propagation.
Unfortunately, the way the code is structured, the ability to construct a grid
is directly dependant on the abiliy to load up a corpus - and how a corpus is
loaded is not at all clear.
Part of the difficulty in figuring out how to work with Textgrounder comes from
the (unexpectedly high) reliance on an imperative coding paradigm for a variety of
tasks, including, apparently, loading text into the document storage and
constructing the cell grid.
It is likely that this structuring is a root cause of the difficulty of
separating out the components we would have wanted to use to determine cell
locations.
Unfortunately, aiding a refactoring of the Textgrounder codebase is outside the
scope of our project, so we will have to make do.
Based on advice from Stephen Roller, we've put together a fairly simple tool
that parses Textgrounder logfiles to extract cell boundary listings, and use
them to determine cell assignment for every geolocated document.
While this should work, it is (as Benjamin Wing pointed out) a fragile and
inelegant workaround.
Lacking a clear alternative, we will have to go forward with it as soon as we
manage to actually run Textgrounder.
\par
We have to admit that most of our lack of progress is the result of our own
poor decision-making, inability to understand the systems we work with, and
sheer bungling, especially on Aidan's part, who admits that he should have done
better.
The key mistake we made was not focusing on getting Textgrounder up and running
on TACC as soon as possible; failing to do that lead inevitably to us
attempting to try to run it on a home computer.
It turns out that running the preprocessing scripts to set up Wikipedia are
impractically slow on any home machine; while it may be possible to actually
perform an evaluation (such as replicating the results of \cite{rolleretal:12})
on such a machine, we recommend against even trying.
In the end, hours of work were wasted on figuring out how to set up
Textgrounder's environment correctly (and often having to repeat time consuming
steps after messing up badly), only for us to decide it was not worth the trouble.
\par
At this point, we decided to try to run Textgrounder on the Linguistics
department machines, where we eventually discovered that (contrary to the notes
in Textgrounder's README) the wikipedia corpus is not at all correctly set up
there for running an evaluation.
Since we do not know what exactly is wrong with the setup, even if we had write
access to the relevant directories, we would not be able to fix the problem
ourselves; at this point, we have finally realized that the sensible decision
is to get onto TACC and perform the evaluations there.

