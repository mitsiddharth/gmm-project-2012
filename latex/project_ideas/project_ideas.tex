\documentclass[11pt]{article}
\usepackage{acl2012}
\usepackage{times}
\usepackage{latexsym}
\usepackage{amsmath}
\usepackage{multirow}
\usepackage{url}
\DeclareMathOperator*{\argmax}{arg\,max}
\setlength\titlebox{6.5cm}    % Expanding the titlebox


\title{GMM 2012 Project Ideas}

\author{Aidan Coyne \\
    UT Austin\\
    Computer Science\\
  {\tt coynea90@gmail.com} \\}

\date{}

\begin{document}
\maketitle

\section{Idea: second-order vectors for geolocation}
\cite{schutze98} suggests that second-order co-occurrence representations are
less sparse than first order models.
Such a model might be helpful for geolocation tasks, which have to use the
relatively limited resources of geotagged text.

As explained in \cite{skiles12}, geolocation can be considered an
information-retrieval task; by combining the text of geotagged documents that
fall in the same cells, pseudo-documents can be constructed to represent each
region on the map - location prediction is then pseudo-document retrieval.
A simple model constructs cells by simply dividing the world up into a uniform grid,
but more advanced techniques for determining regions are possible;
for instance, the adaptive grid approach of \cite{adaptivegrid}.
This approach deals with document-density differences across regions by using
kd-trees to divide the world into varying-sized cells; the model works to
obtain evenly-sized pseudo-documents over all the cells. This adds other
parameters, such as bucket-size cutoff, to already present geolocation
parameters (definitions of region center and document similarity, among others).

These geolocation-as-IR models use first-order frequency to build their
pseudo-document representations. One possible way to apply the idea of a
second-order model would be to build both the first-order geographical model
and a first-order lexical model over the geotagged training set, and then
construct each second-order pseudo-document by using the weights from the first
order geographic model to combine the appropriate lexical vectors. 
The produced model will have the same format as the first-order model, so the
evaluation techniques of \cite{skiles12} and \cite{adaptivegrid} can still be
applied.
Of course, applying such a second order model introduces a further set of
parameters for tuning, such as the application of frequency-modifications
(e.g. tf-idf) at any of the stages in producing the model.
To avoid having to test all the combinations of the text model parameters and
geographic model parameters, it would make sense to compare to the tunings of
previous work wherever possible.

\section{Label-propogation on Wikipedia}
An idea related to the above was discussed in class; it involves using label
propogation to push locational information into non-geotagged wikipedia
articles.
The labels for propogation would be cells over the globe.
The text of each article would then be added to each cell, for use by
TextGrounder, weighted by the cell's weight on the article.
This would take advantage of the structural properties of wikipedia to extend 
the application of geolocation to information in articles outside of the
geotagged set, and like the previous idea, might reduce the sparseness of the
geolocation model advantageously.

\bibliographystyle{acl2012}
\bibliography{refs}
\end{document}
