\documentclass[11pt]{article}
\usepackage{acl2012}
\usepackage{times}
\usepackage{latexsym}
\usepackage{amsmath}
\usepackage{multirow}
\usepackage{url}
\DeclareMathOperator*{\argmax}{arg\,max}
\setlength\titlebox{6.5cm}    % Expanding the titlebox


\title{GMM 2012 Project Proposal}

\author{Aidan Coyne \\
    UT Austin\\
    Computer Science\\
  {\tt coynea90@gmail.com} \\ 
  \And
  Prateek Maheshwari\\
  UT Austin\\} %todo fill this out

\date{}

\begin{document}
\maketitle

\section{Introduction}
%motivation - why do we want to do this?
%well, so that we can investigate the geolocation thing on the much larger non-geotagged section of wikipedia!
%of course, this isn't straightforward
Prior work applying geolocation models to Wikipedia has been limited to a small portion of its articles;
this is simply because most Wikipedia articles do not have geolocation tags.
This should not be suprising, since most articles on wikipedia are not about specific locations, or objects at specific locations.
However, there are plenty of articles that may be associated with multiple places; for example, biographical articles may refer to the various locations the subject has lived.
Using label propogation, it may be possible to get sensible locations associated with the larger set of unlocated Wikipedia articles.

\section{Related Work}
\par Textgrounder:
\cite{rolleretal:12}

\par Junto:
\cite{talukdar:09} \cite{talukdar:10}

\section{Plans}
\par %simple eval
How does label propagation perform in comparison to Textgrounder?
In order to evaluate this, it can be tested on the labeled set of articles the same way as textgrounder is tested -
labels are left off for a test set, label propogation is run, then the predicted labels are compared to the actual locations of the test set.

\par issues:\\
-predicted locations from label propogation\\
-what are the labels going to be? uniform bins vs. kd-trees\\
-extending to full set.


what other things do we want to try?

\bibliographystyle{acl2012}
\bibliography{refs}
\end{document}
