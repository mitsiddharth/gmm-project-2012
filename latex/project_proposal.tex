\documentclass[11pt]{article}
\usepackage{acl2012}
\usepackage{times}
\usepackage{latexsym}
\usepackage{amsmath}
\usepackage{multirow}
\usepackage{url}
\DeclareMathOperator*{\argmax}{arg\,max}
\setlength\titlebox{6.5cm}    % Expanding the titlebox


\title{Geolocating Wikipedia Articles Using Label Propagation}

\author{Aidan Coyne \\
    Department of Computer Science\\
    The University of Texas at Austin\\
  {\tt coynea90@gmail.com} \\ 
  \And
  Prateek Maheshwari\\
     Department of Computer Science\\
    The University of Texas at Austin\\
  {\tt prateekm@utexas.edu}}

\date{}

\begin{document}
\maketitle

\section{Introduction}
%motivation - why do we want to do this?
%well, so that we can investigate the geolocation thing on the much larger non-geotagged section of wikipedia!
%of course, this isn't straightforward
Prior work applying geolocation models to Wikipedia articles 
% \cite{wing and baldridge} \cite{rolleretal:12} \cite {efran et al}
has been limited to a small subset of Wikipedia articles;
this is simply because most Wikipedia articles do not have geolocation tags.
This should not be surprising, since most articles on wikipedia are not about specific locations, 
or objects at specific locations.
However, there are plenty of articles that may be associated with multiple places; 
for example, biographical articles may refer to the various locations the subject has lived.
Using label propagation, it may be possible to get sensible locations associated with the larger set of non-geolocated Wikipedia articles.
Our hypothesis is that we can infer the location for non-geotagged articles 
by using the location information present in other geotagged articles that they link to. 
Unlike prior work, this approach makes use of the actual content of the articles in addition to the metadata. 
We propose to infer the location using label propagation.
We also propose that these additional geolocatable articles can help improve the language models used for geo-locating other articles.

\section{Related Work}
%Geolocation
Geolocation has been implemented as an information-retrieval task, as in
\cite{skiles:12}; IR-based implementations divide the world up into cells, each
of which are considered to be a pseudo-document consisting of text from every
document appearing in that cell.
The simplest approach is to divide the world into uniformly-sized cells % cite{wing and baldridge}
, but \cite{rolleretal:12} use kd-trees to build cells with uniform amounts of text.
This adaptive grid approach is available in the Textgrounder package.

%something on label propogation
\par
\cite{talukdar:09} \cite{talukdar:10}

\section{Plans}
\par
Various options will have to be examined in setting up the label propagation system.
The labels being propagated will have to represent regions of the world, using
the same bucket principle as IR-based geolocation; the world can be cut up
using uniform cells, or using the kd-trees based adaptive grid of
\cite{rolleretal:12}, among other conceivable methods, though we only intend to
explore those two.
Another source of options comes from defining the location prediction output
from running label propagation.
The most obvious option for this is to use the highest-valued label as the predicted cell;
as for the location of that cell, \cite{rolleretal:12} found that the centroid
of document locations in each cell tended to get the best performance.
Another possibility that might be worth examining is to take the top $k$ cell
locations and output their centroid.
Other issues involve the structure of the graph; for instance, we will have to
determine if a directed or undirected graph is more suitable.
Another question is how the density of the graph will affect the performance of
the label propagation system. % this needs expanding, I don't really get this bit. -Aidan
\par

\section{Evaluation}
In order to explore these options, it will be helpful to have something to test against;
the geolocation task itself can be employed.
Applying a label-propagation model to geolocation should be fairly straightforward; 
provide known correct labeling for the nodes in the training set, run the label
propagation system, and examine the labeling of the nodes in the test set.
To score the models, we will use measures based on error distances.
\par
The motivation for all of this is to extend geolocation to the vast unlabeled
section of wikipedia; with two different models, each with a variety of
parameters, there are multiple ways to go about this.
One scheme would be to use the geolocated portion to train Textgrounder and
compare its predictions for the unlabeled set to the predictions yielded by
running label propagation over the combined sets.

\bibliographystyle{acl2012}
\bibliography{refs}
\end{document}
